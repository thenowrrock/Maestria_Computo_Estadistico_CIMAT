\documentclass{beamer}
\usepackage[utf8]{inputenc}
\usetheme[background=dark,titleformat = smallcaps , block = fill,numbering = fraction, progressbar = 
frametitle , titleformat title= smallcaps]{metropolis}           % Use metropolis theme


\definecolor{orangeBar}{HTML}{FF3600}
\setbeamercolor{progress bar}{fg=orangeBar}

\usepackage{multimedia}
\usepackage{animate}
\usepackage {extarrows}
\usepackage {tikz}
\usepackage[options ]{algorithm2e}
\usepackage[spanish]{babel}
\usepackage{graphicx}
\usepackage{amssymb}
%\usepackage{amsfonts}

\usepackage{enumerate}
\usepackage{amsmath}
\usepackage{amsthm}
\usepackage{xcolor}
%\usepackage{amsfonts,amssymb,amsthm}

\usepackage{url}
\usepackage{enumerate}
\usepackage{commath}
\usepackage{multicol}
\usepackage{mathtools}
\usepackage{scrextend}
\usepackage{hyperref}
\usepackage{cleveref}
\usepackage{longtable}
\usepackage{bbm}
\usepackage{siunitx}
\usepackage{listings}
\usepackage{xcolor}
\usepackage{subcaption}
\usepackage{epigraph}
\usetikzlibrary{arrows}
\usepackage[linesnumbered, ruled, vlined]{algorithm2e}

\definecolor{codegreen}{rgb}{0,0.6,0}
\definecolor{codegray}{rgb}{0.5,0.5,0.5}
\definecolor{codepurple}{rgb}{0.58,0,0.82}
\definecolor{backcolour}{rgb}{0.95,0.95,0.92}
\definecolor{darkBlue}{HTML}{00000F}
\definecolor{lightBlue}{HTML}{00B7D4}
\definecolor{lightRed}{HTML}{E42525}
\definecolor{lightGreen}{HTML}{9CE425}

\definecolor{green0}{HTML}{B65900}
\definecolor{green1}{HTML}{D77200}
\definecolor{green2}{HTML}{ED8E30}
\definecolor{green3}{HTML}{FABE86}

\lstdefinestyle{mystyle}{
    backgroundcolor=\color{darkBlue},   
	commentstyle=\color{lightGreen},
	keywordstyle=\color{lightBlue},
	numberstyle=\tiny\color{codegray},
	stringstyle=\color{lightRed},
	basicstyle=\ttfamily\footnotesize,
	breakatwhitespace=false,         
	breaklines=true,                 
	captionpos=b,                    
	keepspaces=true,                 
	numbers=left,                    
	numbersep=5pt,                  
	showspaces=false,                
	showstringspaces=false,
	showtabs=false,                  
	tabsize=1
}

\lstset{style=mystyle}
\lstset{language=Python}
\lstset{frame=lines}
\lstset{caption={Insert code directly in your document}}
\lstset{label={lst:code_direct}}
\lstset{basicstyle=\footnotesize}


\newcommand{\bb}[1]{\mathbb{#1}}

%\newtheorem{theorem}{Teorema}[section]
%\theoremstyle{plain}
\newtheorem{acknowledgement}[theorem]{Acknowledgement}
%\newtheorem{algorithm}[theorem]{Algorithm}
\newtheorem{axiom}[theorem]{Axiom}
\newtheorem{case}[theorem]{Case}
\newtheorem{claim}{Claim}
\newtheorem{conclution}[theorem]{Conclusión}
\newtheorem{condition}[theorem]{Condition}
\newtheorem{conjecture}[theorem]{Conjecture}
%\newtheorem{corollary}[theorem]{Corolario}
\newtheorem{criterion}[theorem]{Criterion}
\theoremstyle{definition}
%\newtheorem*{df}{Definición}
%\newtheorem{definition}[theorem]{Definición}
%\newtheorem{example}[theorem]{Ejemplo}
\newtheorem{exercise}[theorem]{Exercise}
%\newtheorem{lemma}[theorem]{Lema}
\newtheorem{notation}[theorem]{Notation}
%\newtheorem{problem}[theorem]{Problem}
\newtheorem{proposition}[theorem]{Proposición}
\newtheorem{remark}[theorem]{Nota}
%\newtheorem{solution}[theorem]{Solución}
\newtheorem{summary}[theorem]{Summary}
\numberwithin{equation}{section}

\definecolor{defColor}{HTML}{3ED597}
\newcommand{\marine}[1]{\textcolor{defColor}{#1}}


\definecolor{thColor}{HTML}{FA7E0A}
\newcommand{\orangee}[1]{\textcolor{thColor}{#1}}

\definecolor{rkColor}{HTML}{F72121}
\newcommand{\redd}[1]{\textcolor{rkColor}{#1}}


%---------------emojis--------------------
\newcommand{\smiley}{\tikz[baseline=-0.75ex,black]{
		\draw circle (2mm);
		\node[fill,circle,inner sep=0.5pt] (left eye) at (135:0.8mm) {};
		\node[fill,circle,inner sep=0.5pt] (right eye) at (45:0.8mm) {};
		\draw (-145:0.9mm) arc (-120:-60:1.5mm);
	}
}

\newcommand{\frownie}{\tikz[baseline=-0.75ex,black]{
		\draw circle (2mm);
		\node[fill,circle,inner sep=0.5pt] (left eye) at (135:0.8mm) {};
		\node[fill,circle,inner sep=0.5pt] (right eye) at (45:0.8mm) {};
		\draw (-145:0.9mm) arc (120:60:1.5mm);
	}
}

\newcommand{\neutranie}{\tikz[baseline=-0.75ex,black]{
		\draw circle (2mm);
		\node[fill,circle,inner sep=0.5pt] (left eye) at (135:0.8mm) {};
		\node[fill,circle,inner sep=0.5pt] (right eye) at (45:0.8mm) {};
		\draw (-135:0.9mm) -- (-45:0.9mm);
	}
}




\newtheorem{df}{\marine{Definición}}
\newtheorem{thh}{\orangee{Teorema}}
\newtheorem{cod}{\orangee{Código}}
\newtheorem{pr}{\orangee{Proposición}}
\newtheorem{lm}{\orangee{Lema}}
\newtheorem{crr}{\orangee{Corolario}}
\newtheorem{rr}{\orangee{Observación}}
\newtheorem{algor}{\orangee{}}

\usepackage{graphicx} 


%\newtheorem{defn}[]{Definición}
%\newenvironment{definition}{\begin{defn}}{\end{defn}}
%\newtheorem{definition}{Definition}[section]
%\newtheorem*{remark}{Remark}
%%%%%%%%
\newcommand{\tit}[1]{\textit{#1}}
\newcommand{\bsym}{\mathbf}
\newcommand{\Mod}[1]{\ (\mathrm{mod}\ #1)}
%\newcommand{\blue}[1]{\textcolor{blue}{#1}}
\newcommand{\red}[1]{\textcolor{red}{#1}}
\renewcommand{\geq}{\geqslant}
\renewcommand{\leq}{\leqslant}
\newcommand{\Rplus}{\mathds{R}_{^{+}}}
\newcommand{\N}{\mathbb{N}}
\newcommand{\Z}{\mathbb{Z}}
\newcommand{\R}{\mathbb{R}}

\newcommand{\C}{\mathbb{C}}
\newcommand{\Q}{\mathbb{Q}}
\newcommand{\ssi}{\longleftrightarrow}
\newcommand{\ent}{\longrightarrow}
\newcommand{\Qp}{\mathbb{Q}_p}  
\newcommand{\Qpn}{\mathbb{Q}_p^n}
\newcommand{\Zpn}{\mathbb{Z}_p^n}
\newcommand{\Zp}{\mathbb{Z}_p}
\newcommand{\Zd}{\mathbb{Z}_2}
%\newcommand{\abs}[1]{\left\vert #1 \right\vert}
%\newcommand{\norm}[1]{\|#1\|}
\newcommand{\pnorm}[1]{\|#1\|_p}
\newcommand{\maxx}[1]{\text{m\'ax} #1}
\newcommand{\xbar}[1]{\hskip 1.4pt\overline{\hskip-1.2pt #1\hskip -.6pt}\hskip 1.2pt}
\newcommand{\rb}{\raisebox{-.35ex}}

\DeclareMathOperator{\s}{\mathbf{S}}
\DeclareMathOperator{\f}{\mathcal{F}}
\DeclareMathOperator{\A}{\mathbb{A}}
\DeclareMathOperator{\dist}{dist} % The distance.
\DeclareMathOperator{\d^n}{\dif^{\,n}}
%\DeclareMathOperator{\d}{\dif}
\DeclareMathOperator{\Real}{Re}
\DeclareMathOperator{\ord}{Ord}
\DeclareMathOperator{\Dom}{Dom}
\DeclareMathOperator{\vol}{vol}
\DeclareMathOperator{\gpn}{\mathit{{GpnN}}}
%%%%%%%%%
%%%%%%%%%%%%%%%%% dashed integrals %%%%%%%%%%%%%%%%%%%%%
\DeclareSymbolFont{eulargesymbols}{U}{zeuex}{m}{n}
\DeclareMathSymbol{\intop}{\mathop}{eulargesymbols}{"52}
\usepackage[toc,page]{appendix}
\renewcommand{\labelitemi}{$\circ$}


\title{Regularización en métodos de regresión}
\subtitle{Proyecto final, Ciencia de datos}
\date{02 de Junio de 2021}


\author{\bf{Autores: }Enrique Santibáñez}

\institute{Centro de Investigación en Matemáticas, \\Maestría en Cómputo Estadístico.}
%Facultad de Ciencias\\
%Departamento de Matemáticas}

\newcommand{\X}{\mathbf{X}}
\newcommand{\x}{\mathbf{x}}
\newcommand{\Y}{\mathbf{Y}}
\newcommand{\y}{\mathbf{y}}
\newcommand{\Z}{\mathbf{Z}}
\newcommand{\E}{\mathbf{E}}
\newcommand{\B}{\mathbf{B}}

\usepackage{MnSymbol,wasysym}



\titlegraphic{%
	\begin{picture}(0,0)
	\put(190,-190){\makebox(0,0)[rt]{\includegraphics[width=2cm]{logo}}}
    \end{picture}
}

\usepackage[backend=biber, style=apa, natbib = true]{biblatex}
\bibliography{biblio.bib}