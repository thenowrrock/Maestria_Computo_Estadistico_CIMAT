\documentclass[11pt,letterpaper]{article}
\usepackage[utf8]{inputenc}
\usepackage[T1]{fontenc}
\usepackage[spanish]{babel}
\usepackage{amsmath}
\usepackage{amsfonts}
\usepackage{amssymb}
\usepackage{graphicx}
\usepackage{lmodern}
\usepackage{xspace}
\usepackage{multicol}
\usepackage{hyperref}
\usepackage{float}
\usepackage{hyperref}
\usepackage{color}

\newcommand{\azul}[1]{\textcolor{MaterialBlue900}{#1}}
\usepackage{array}

\hypersetup{colorlinks=true,   linkcolor=MaterialBlue900}
%\usepackage[colorlinks=true, linkcolor=black, urlcolor=blue, pdfborder={0 0 0}]{hyperref}

\usepackage[left=2cm,right=2cm,top=2cm,bottom=2cm]{geometry}
\title{Modelos no paramétricos y de regresión 2018-1}
\author{Tarea examen: pruebas binomiales y tablas de contingencia}
\date{Fecha de entrega: 08/01/2017}
\setlength{\parindent}{0in}
\spanishdecimal{.}


\newcommand{\X}{\mathbb{X}}
\newcommand{\x}{\mathbf{x}}
\newcommand{\Y}{\mathbf{Y}}
\newcommand{\y}{\mathbf{y}}
\newcommand{\xbarn}{\bar{x}_n}
\newcommand{\ybarn}{\bar{y}_n}
\newcommand{\paren}[1]{\left( #1 \right)}
\newcommand{\llaves}[1]{\left\lbrace #1 \right\rbrace}
\newcommand{\barra}{\,\vert\,}
\newcommand{\mP}{\mathbb{P}}
\newcommand{\mE}{\mathbb{E}}
\newcommand{\mI}{\mathbf{I}}
\newcommand{\mJ}{\mathbf{J}}
\newcommand{\mX}{\mathbf{X}}
\newcommand{\mS}{\mathbf{S}}
\newcommand{\mA}{\mathbf{A}}
\newcommand{\unos}{\boldsymbol{1}}
\newcommand{\xbarnv}{\bar{\mathbf{x}}_n}
\newcommand{\abs}[1]{\left\vert #1 \right\vert}
\newcommand{\muv}{\boldsymbol{\mu}}
\newcommand{\mcov}{\boldsymbol{\Sigma}}
\newcommand{\vbet}{\boldsymbol{\beta}}
\newcommand{\veps}{\boldsymbol{\epsilon}}
\newcommand{\mC}{\mathbf{C}}
\newcommand{\ceros}{\boldsymbol{0}}
\newcommand{\mH}{\mathbf{H}}
\newcommand{\ve}{\mathbf{e}}
\newcommand{\avec}{\mathbf{a}}
\newcommand{\res}{\textbf{RESPUESTA}\\}
\newcommand{\rojo}[1]{\textcolor{MaterialRed900}{#1}}

\newcommand{\defi}[3]{\textbf{Definición:#3}}
\newcommand{\fin}{$\blacksquare.$}
\newcommand{\finf}{\blacksquare.}

\begin{document}
\begin{table}[ht]
\centering
\begin{tabular}{c}
\textbf{Maestría en Computo Estadístico}\\
\textbf{Álgebra Matricial} \\
\textbf{Tarea 1}\\
\today \\
\emph{Enrique Santibáñez Cortés}\\
Repositorio de Git: \href{https://github.com/Enriquesec/Inferencia_Estad-stica/tree/master/Tareas/Tarea_1}{Tarea 1, IE}.
\end{tabular}
\end{table}

\begin{enumerate}
\item Si $A$ es una matriz $m\times n$ dada por bloques de vectores columna como $$ (a_1\ a_2\ \cdots a_n)$$ y $B$ es una matriz $n\times p$ dada por bloques de vectores renglón como
\begin{equation*}
\left(\begin{array}{c}
v1\\
v2\\
\vdots \\
v_n
\end{array}
\right)
\end{equation*}
Demuestre que $$AB=\sum_{i=1}^na_iv_i.$$

\item Sean $A$ y $B$ matrices cuadradas del mismo orden. Demuestre que $(A-B)(A+B)=A^2-B^2$ si y solo si $AB=BA.$

\item Sean $A$ y $B$ matrices $n\times n$, $A\neq 0, \ B\neq 0,$ tales que $AB=BA.$ Demuestre que $A^pB^p=B^pA^p$ para cualesquiera $p,q\in \mathbb{N}.$

\item Se dice que una matriz cuadrada $A$ es antisimétrica si $A=-A^t.$ Demuestre que $A-A^t$ es antisimétrica.

\item Demuestre que dada cualquier matriz cuadrada $A,$ esta se puede escribir como la suma de una matriz simétrica y una matriz antisimétrica.

\item Se dice que una matriz cuadrada $P$ es idempotente si $P^2=P$. Si 
\begin{equation*}
A=\left(\begin{array}{cc}
I& P\\
0&P
\end{array}
\right)
\end{equation*}
y si $P$ es idempotente, encuentre $A^{500}.$

\item Sean $A$ y $B$ matrices de tamaño $m\times n$. Demuestre que tr$(AB^t)=$tr$(A^tB)$. 

\item Encuentre matrices $A, B \ \text{y} \ C$ tales que tr$(ABC)\neq$tr$(BAC)$.

\item Sea $L$ una mtriz triangular inferior $n\times n$. Demuestre que $L=L_1L_2\cdots L_n$ donde $L_i$ es la matriz $n\times n$ que se obtiene reemplazando la $i-$ésima columna de $I_n$ por la $i-$ésima columna de $L$. Demuestre un resultado análogo para matrices triangulares superiores.

\item Sea $A=(a_{ij})$ una matriz cuadrada de tamaño $n$ tal que $a_{ij}=0$ para $i=1, \cdots , n.$ Demuestre que para $i=1, \cdots , n$ y $j=1, \cdots , \min (n,i+p-1)$ se cumple que $b_{ij}=0$ donde $A^p=(b_{ij})$ y $p$ es un entero positivo. 



\end{enumerate}
\end{document}