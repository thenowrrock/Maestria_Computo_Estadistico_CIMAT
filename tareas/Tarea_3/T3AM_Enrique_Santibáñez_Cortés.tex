\documentclass[11pt,letterpaper]{article}
\usepackage[utf8]{inputenc}
\usepackage[T1]{fontenc}
\usepackage[spanish]{babel}
\usepackage{amsmath}
\usepackage{amsfonts}
\usepackage{amssymb}
\usepackage{graphicx}
\usepackage{lmodern}
\usepackage{xspace}
\usepackage{multicol}
\usepackage{hyperref}
\usepackage{float}
\usepackage{hyperref}
\usepackage{color}

\newcommand{\azul}[1]{\textcolor{MaterialBlue900}{#1}}
\usepackage{array}

\hypersetup{colorlinks=true,   linkcolor=MaterialBlue900}
%\usepackage[colorlinks=true, linkcolor=black, urlcolor=blue, pdfborder={0 0 0}]{hyperref}

\usepackage[left=2cm,right=2cm,top=2cm,bottom=2cm]{geometry}
\title{Modelos no paramétricos y de regresión 2018-1}
\author{Tarea examen: pruebas binomiales y tablas de contingencia}
\date{Fecha de entrega: 08/01/2017}
\setlength{\parindent}{0in}
\spanishdecimal{.}


\newcommand{\X}{\mathbb{X}}
\newcommand{\x}{\mathbf{x}}
\newcommand{\Y}{\mathbf{Y}}
\newcommand{\y}{\mathbf{y}}
\newcommand{\xbarn}{\bar{x}_n}
\newcommand{\ybarn}{\bar{y}_n}
\newcommand{\paren}[1]{\left( #1 \right)}
\newcommand{\llaves}[1]{\left\lbrace #1 \right\rbrace}
\newcommand{\barra}{\,\vert\,}
\newcommand{\mP}{\mathbb{P}}
\newcommand{\mE}{\mathbb{E}}
\newcommand{\mI}{\mathbf{I}}
\newcommand{\mJ}{\mathbf{J}}
\newcommand{\mX}{\mathbf{X}}
\newcommand{\mS}{\mathbf{S}}
\newcommand{\mA}{\mathbf{A}}
\newcommand{\unos}{\boldsymbol{1}}
\newcommand{\xbarnv}{\bar{\mathbf{x}}_n}
\newcommand{\abs}[1]{\left\vert #1 \right\vert}
\newcommand{\muv}{\boldsymbol{\mu}}
\newcommand{\mcov}{\boldsymbol{\Sigma}}
\newcommand{\vbet}{\boldsymbol{\beta}}
\newcommand{\veps}{\boldsymbol{\epsilon}}
\newcommand{\mC}{\mathbf{C}}
\newcommand{\ceros}{\boldsymbol{0}}
\newcommand{\mH}{\mathbf{H}}
\newcommand{\ve}{\mathbf{e}}
\newcommand{\avec}{\mathbf{a}}
\newcommand{\res}{\textbf{RESPUESTA}\\}
\newcommand{\rojo}[1]{\textcolor{MaterialRed900}{#1}}

\newcommand{\defi}[3]{\textbf{Definición:#3}}
\newcommand{\fin}{$\blacksquare.$}
\newcommand{\finf}{\blacksquare.}
\newcommand{\tr}{\text{tr}}

\begin{document}
\begin{table}[ht]
\centering
\begin{tabular}{c}
\textbf{Maestría en Computo Estadístico}\\
\textbf{Álgebra Matricial} \\
\textbf{Tarea 3}\\
\today \\
\emph{Enrique Santibáñez Cortés}\\
Repositorio de Git: \href{https://github.com/Enriquesec/Algebra_matricial/tree/feature/tareas/tareas/Tarea_3}{Tarea 3, AM}.
\end{tabular}
\end{table}
Todos los cálculos deben ser a mano.

\begin{enumerate}
% Problema 1.
%------------------------------------------------------------------------------
%------------------------------------------------------------------------------
\item Dada la matriz
\begin{equation*}
\left( \begin{array}{rrrr}
-4 & 5 & -6 &7\\
-1 & 1 &  1 &3\\
1  & 2 & -3 &-1
\end{array} \right)
\end{equation*}
encuentre su forma escalonada reducida por renglones. Escriba todas las matrices elementales correspondientes a las operaciones que usó para llevar la matriz a la forma que obtuvo.

% Problema 2.
%------------------------------------------------------------------------------
%------------------------------------------------------------------------------
\item Dada el sistema $Ax=b$, donde 
\begin{equation*}
A=\left( \begin{array}{rrr}
1 & 3 & -1\\
a_1 & -1 & -3\\
1 & 2 &2
\end{array} \right) \ \ \ \text{y} \ \ \ b=\left(\begin{array}{c}
0\\
1\\
a_2
\end{array} \right)
\end{equation*}
encuentre condiciones generales sobre $a_1$ y $a_2$ para que el sistema sea consistente. Si se quiere que la solución sea exactamente $x=(3,-1,2)^t$ , ¿qué valores deben
tener $a_1$ y $a_2$?

% Problema 3.
%------------------------------------------------------------------------------
%------------------------------------------------------------------------------
\item Encuentre la solución general, escribiéndola como combinación lineal de vectores, del sistema homogéneo $Ax=0$ donde

\begin{equation*}
\left(\begin{array}{rrrrrr}
1  & -3 & 1 & -1 &  0 & 1\\
-1 &  3 & 0 &  3 &  1 & 3 \\
2  & -6 & 3 &  0 & -1 & 2\\
-1 &  3 & 1 &  5 &  1 & 6
\end{array}\right).
\end{equation*}

% Problema 4.
%------------------------------------------------------------------------------
%------------------------------------------------------------------------------
\item encuentra la inversa de 
\begin{equation*}
\left(\begin{array}{rrr}
 1 &  0 & -2\\
-3 &  1 &  4\\
 2 & -3 &  4
\end{array}\right).
\end{equation*}


% Problema 5.
%------------------------------------------------------------------------------
%------------------------------------------------------------------------------
\item Sea 
\begin{equation*}
A=\left(\begin{array}{rrr}
 1 & 1 & 1\\
 0 & 2 & 3\\
 5 & 5 & 1
\end{array}\right).
\end{equation*}
Demuestre que $A$ es no singular y luego escriba $A$ como producto de matrices elementales.

% Problema 6.
%------------------------------------------------------------------------------
%------------------------------------------------------------------------------
\item i) Encuentre dos matrices que sean invertibles pero que su suma no sea invertible. ii) Encuentre dos matrices singulares cuya suma sea invertible. Justifique todas sus aseveraciones.


% Problema 7.
%------------------------------------------------------------------------------
%------------------------------------------------------------------------------
\item Encuentre la descomposición LU de la matriz
\begin{equation*}
\left(\begin{array}{rrrr}
 1 &  2 & -1 & 4\\
 0 & -1 &  5 & 8\\
 2 &  3 &  1 & 4\\
 1 & -1 &  6 & 4
\end{array} \right).
\end{equation*}

% Problema 8.
%------------------------------------------------------------------------------
%------------------------------------------------------------------------------
\item Encuentre la descomposición LU de la matriz
\begin{equation*}
A=\left(\begin{array}{rrrr}
 2 &  3 & -1 & 6\\
 4 &  7 &  2 & 1\\
-2 &  5 & -2 & 0\\
 0 & -4 &  5 & 2
\end{array} \right),
\end{equation*}
y luego úsela para encontrar la solución del sistema $Ax=b$, donde 
\begin{equation*}
b=\left(\begin{array}{c}
1\\
0\\
0\\
4
\end{array} \right).
\end{equation*}
% Problema 9.
%------------------------------------------------------------------------------
%------------------------------------------------------------------------------
\item Encuentre la descomposición LU de la matriz 
\begin{equation*}
A=\left(\begin{array}{rrrr}
 1 & -2 & -2 &-3\\
 3 & -9 &  0 &-9\\
-1 &  2 &  4 & 7\\
-3 & -6 & 26 & 2
\end{array} \right),
\end{equation*}
Usando esta misma descomposición como ayuda, encuentre $A^{-1}$.


% Problema 10.
%------------------------------------------------------------------------------
%------------------------------------------------------------------------------
\item Encuentre la descomposición LU de la matriz por bandas
\begin{equation*}
A=\left(\begin{array}{cccc}
a_{11} & a_{12} &   0    &    0  \\
a_{21} & a_{22} & a_{23} &    0  \\
    0  & a_{32} & a_{33} & a_{34}\\
    0  &   0    & a_{43} & a_{44}
\end{array} \right),
\end{equation*}
(Para una interesante aplicación de matrices por bandas a problemas de flujo de calor en física y la importancia de obtener su descomposición LU, ver problemas 31 y 32 de Linear Algebra, D. Lay, 4th ed., p. 131 y las explicaciones que ahí se dan.)

\end{enumerate}
\end{document}