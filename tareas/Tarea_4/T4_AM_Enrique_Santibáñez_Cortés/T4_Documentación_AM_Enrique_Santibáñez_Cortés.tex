\documentclass[11pt,letterpaper]{article}
\usepackage[utf8]{inputenc}
\usepackage[T1]{fontenc}
\usepackage[spanish]{babel}
\usepackage{amsmath}
\usepackage{amsfonts}
\usepackage{amssymb}
\usepackage{graphicx}
\usepackage{lmodern}
\usepackage{xspace}
\usepackage{multicol}
\usepackage{hyperref}
\usepackage{float}
\usepackage{hyperref}
\usepackage{color}

\newcommand{\azul}[1]{\textcolor{MaterialBlue900}{#1}}
\usepackage{array}

\hypersetup{colorlinks=true,   linkcolor=MaterialBlue900}
%\usepackage[colorlinks=true, linkcolor=black, urlcolor=blue, pdfborder={0 0 0}]{hyperref}

\usepackage[left=2cm,right=2cm,top=2cm,bottom=2cm]{geometry}
\title{Modelos no paramétricos y de regresión 2018-1}
\author{Tarea examen: pruebas binomiales y tablas de contingencia}
\date{Fecha de entrega: 08/01/2017}
\setlength{\parindent}{0in}
\spanishdecimal{.}


\newcommand{\X}{\mathbb{X}}
\newcommand{\x}{\mathbf{x}}
\newcommand{\Y}{\mathbf{Y}}
\newcommand{\y}{\mathbf{y}}
\newcommand{\xbarn}{\bar{x}_n}
\newcommand{\ybarn}{\bar{y}_n}
\newcommand{\paren}[1]{\left( #1 \right)}
\newcommand{\llaves}[1]{\left\lbrace #1 \right\rbrace}
\newcommand{\barra}{\,\vert\,}
\newcommand{\mP}{\mathbb{P}}
\newcommand{\mE}{\mathbb{E}}
\newcommand{\mI}{\mathbf{I}}
\newcommand{\mJ}{\mathbf{J}}
\newcommand{\mX}{\mathbf{X}}
\newcommand{\mS}{\mathbf{S}}
\newcommand{\mA}{\mathbf{A}}
\newcommand{\unos}{\boldsymbol{1}}
\newcommand{\xbarnv}{\bar{\mathbf{x}}_n}
\newcommand{\abs}[1]{\left\vert #1 \right\vert}
\newcommand{\muv}{\boldsymbol{\mu}}
\newcommand{\mcov}{\boldsymbol{\Sigma}}
\newcommand{\vbet}{\boldsymbol{\beta}}
\newcommand{\veps}{\boldsymbol{\epsilon}}
\newcommand{\mC}{\mathbf{C}}
\newcommand{\ceros}{\boldsymbol{0}}
\newcommand{\mH}{\mathbf{H}}
\newcommand{\ve}{\mathbf{e}}
\newcommand{\avec}{\mathbf{a}}
\newcommand{\res}{\textbf{RESPUESTA}\\}
\newcommand{\rojo}[1]{\textcolor{MaterialRed900}{#1}}

\newcommand{\defi}[3]{\textbf{Definición:#3}}
\newcommand{\fin}{$\blacksquare.$}
\newcommand{\finf}{\blacksquare.}
\newcommand{\tr}{\text{tr}}
\begin{document}
\begin{table}[ht]
\centering
\begin{tabular}{c}
\textbf{Maestría en Computo Estadístico}\\
\textbf{Álgebra Matricial} \\
\textbf{Tarea 4}\\
\today \\
\emph{Enrique Santibáñez Cortés}\\
Repositorio de Git: \href{https://github.com/Enriquesec/Algebra_matricial/tree/master/tareas/Tarea_4/}{Tarea 4, AM}.
\end{tabular}
\end{table}
Este documento tiene la finalidad de explicar el uso de las funciones de la tarea 4 de Algébra Matricial. 

\begin{enumerate}
\item La primera función cumple con las siguientes especificaciones.

\textit{Programe una función en r que reciba de entrada una matriz cuadrada y regrese la descomposición LU de la matriz. Si esta no se puede obtener, debe regresar un mensaje que diga por qué y donde fue donde estuvo el problema.}

La función que realiza lo anterior esta en el archivo en R:

\textit{descomposicion\_LU\_T4\_Enrique\_Santibáñez\_Cortes.r}

\begin{itemize}
\item[Paso 1.] Correr todo el script completo.

\item[Paso 2.] Se solicitará el número de renglones de la matriz a ingresar. Este paso tiene una validación para solo admitir número enteros positivos, por lo que no podrá pasar al siguiente si no cumple esa condición. Dar enter para continuar. Ejemplo: Se ingresa un 2.
\begin{figure}[H]
\centering
\includegraphics[scale=.7]{paso_2.png}
\end{figure}

\item[Paso 3.] Solicitará los elementos de izquierda a derecha, y arriba a abajo de la matriz. Se valida que solo ingrese números, de caso contrario solicitará de nuevo la entrada. Ejemplo: se agrega la matriz con los elementos: $\begin{pmatrix}
1&2\\
3&3
\end{pmatrix}$.
\begin{figure}[H]
\centering
\includegraphics[scale=.7]{paso_3.png}
\end{figure}

\item[Resultados] Se imprimirá la descomposición $LU$ de la matriz (si es que existe), primero muestra la matriz $U$ y después la matriz $L$. Ejemplo:
\begin{figure}[H]
\centering
\includegraphics[scale=.7]{LU_resultados.png}
\end{figure}

\item[Notas] Si la matriz ingresado no tiene descomposición muestra el error por la cual no tiene y imprime la última matriz modificada para llegar a $U$. Ejemplo:
\begin{figure}[H]
\centering
\includegraphics[scale=.7]{LU_nota.png}
\end{figure}
\end{itemize}


\item La segunda función cumple con las siguientes especificaciones:

\textit{Programe una función en r que reciba de entrada una matriz cuadrada y regrese la descomposición PA=LU de la matriz. Debe usar el método de pivoteo parcial, por lo que parte de la tarea es investigar como funciona ese método.}

La función que realiza lo anterior esta en el archivo en R:

\textit{descomposicion\_PA\_LU\_T4\_Enrique\_Santibáñez\_Cortes.r}

\begin{itemize}
\item[Paso 1.] Correr todo el script completo.

\item[Paso 2.] Se solicitará el número de renglones de la matriz a ingresar. Este paso tiene una validación para solo admitir número enteros positivos, por lo que no podrá pasar al siguiente si no cumple esa condición. Dar enter para continuar. Ejemplo: Se ingresa un 2.
\begin{figure}[H]
\centering
\includegraphics[scale=.7]{paso_2.png}
\end{figure}

\item[Paso 3.] Solicitará los elementos de izquierda a derecha, y arriba a abajo de la matriz. Se valida que solo ingrese números, de caso contrario solicitará de nuevo la entrada. Ejemplo: se agrega la matriz con los elementos: $\begin{pmatrix}
1&2\\
3&3
\end{pmatrix}$.
\begin{figure}[H]
\centering
\includegraphics[scale=.7]{paso_3.png}
\end{figure}

\item[Resultados] Se imprimirá la descomposición $PA=LU$ de la matriz (si es que existe), primero muestra la matriz $U$, después la matriz $L$ y por último la matriz P. Ejemplo:
\begin{figure}[H]
\centering
\includegraphics[scale=.7]{PA_LU_resultados.png}
\end{figure}

\item[Notas] Si la matriz ingresado no tiene descomposición muestra el error por la cual no tiene y imprime la última matriz modificada para llegar a $U$. Ejemplo:
\begin{figure}[H]
\centering
\includegraphics[scale=.7]{PA_LU_nota.png}
\end{figure}
\end{itemize}

\end{enumerate}

\end{document}