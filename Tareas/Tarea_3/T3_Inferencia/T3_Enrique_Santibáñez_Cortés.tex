\documentclass[11pt,]{article}
\usepackage{lmodern}
\usepackage{amssymb,amsmath}
\usepackage{ifxetex,ifluatex}
\usepackage{fixltx2e} % provides \textsubscript
\ifnum 0\ifxetex 1\fi\ifluatex 1\fi=0 % if pdftex
  \usepackage[T1]{fontenc}
  \usepackage[utf8]{inputenc}
\else % if luatex or xelatex
  \ifxetex
    \usepackage{mathspec}
  \else
    \usepackage{fontspec}
  \fi
  \defaultfontfeatures{Ligatures=TeX,Scale=MatchLowercase}
\fi
% use upquote if available, for straight quotes in verbatim environments
\IfFileExists{upquote.sty}{\usepackage{upquote}}{}
% use microtype if available
\IfFileExists{microtype.sty}{%
\usepackage{microtype}
\UseMicrotypeSet[protrusion]{basicmath} % disable protrusion for tt fonts
}{}
\usepackage[left=2cm,right=2cm,top=2cm,bottom=2cm]{geometry}
\usepackage{hyperref}
\hypersetup{unicode=true,
            pdfborder={0 0 0},
            breaklinks=true}
\urlstyle{same}  % don't use monospace font for urls
\usepackage{graphicx,grffile}
\makeatletter
\def\maxwidth{\ifdim\Gin@nat@width>\linewidth\linewidth\else\Gin@nat@width\fi}
\def\maxheight{\ifdim\Gin@nat@height>\textheight\textheight\else\Gin@nat@height\fi}
\makeatother
% Scale images if necessary, so that they will not overflow the page
% margins by default, and it is still possible to overwrite the defaults
% using explicit options in \includegraphics[width, height, ...]{}
\setkeys{Gin}{width=\maxwidth,height=\maxheight,keepaspectratio}
\IfFileExists{parskip.sty}{%
\usepackage{parskip}
}{% else
\setlength{\parindent}{0pt}
\setlength{\parskip}{6pt plus 2pt minus 1pt}
}
\setlength{\emergencystretch}{3em}  % prevent overfull lines
\providecommand{\tightlist}{%
  \setlength{\itemsep}{0pt}\setlength{\parskip}{0pt}}
\setcounter{secnumdepth}{0}
% Redefines (sub)paragraphs to behave more like sections
\ifx\paragraph\undefined\else
\let\oldparagraph\paragraph
\renewcommand{\paragraph}[1]{\oldparagraph{#1}\mbox{}}
\fi
\ifx\subparagraph\undefined\else
\let\oldsubparagraph\subparagraph
\renewcommand{\subparagraph}[1]{\oldsubparagraph{#1}\mbox{}}
\fi

%%% Use protect on footnotes to avoid problems with footnotes in titles
\let\rmarkdownfootnote\footnote%
\def\footnote{\protect\rmarkdownfootnote}

%%% Change title format to be more compact
\usepackage{titling}

% Create subtitle command for use in maketitle
\newcommand{\subtitle}[1]{
  \posttitle{
    \begin{center}\large#1\end{center}
    }
}

\setlength{\droptitle}{-2em}

  \title{}
    \pretitle{\vspace{\droptitle}}
  \posttitle{}
    \author{}
    \preauthor{}\postauthor{}
    \date{}
    \predate{}\postdate{}
  
\usepackage[utf8]{inputenc}
\usepackage[T1]{fontenc}
\usepackage[spanish]{babel}
\usepackage{amsmath}
\usepackage{amsfonts}
\usepackage{amssymb}
%\usepackage{graphicx}
\usepackage{lmodern}
\usepackage{xspace}
\usepackage{multicol}
\usepackage{hyperref}
\usepackage{float}
\usepackage{hyperref}
\usepackage{color}

\newcommand{\azul}[1]{\textcolor{MaterialBlue900}{#1}}
\usepackage{array}

\hypersetup{colorlinks=true,   linkcolor=MaterialBlue900}
%\usepackage[colorlinks=true, linkcolor=black, urlcolor=blue, pdfborder={0 0 0}]{hyperref}
%\usepackage[left=2cm,right=2cm,top=2cm,bottom=2cm]{geometry}
\setlength{\parindent}{0in}
\spanishdecimal{.}

\newcommand{\X}{\mathbb{X}}
\newcommand{\x}{\mathbf{x}}
\newcommand{\Y}{\mathbf{Y}}
\newcommand{\y}{\mathbf{y}}
\newcommand{\xbarn}{\bar{x}_n}
\newcommand{\ybarn}{\bar{y}_n}
\newcommand{\paren}[1]{\left( #1 \right)}
\newcommand{\llaves}[1]{\left\lbrace #1 \right\rbrace}
\newcommand{\barra}{\,\vert\,}
\newcommand{\mP}{\mathbb{P}}
\newcommand{\mE}{\mathbb{E}}
\newcommand{\mI}{\mathbf{I}}
\newcommand{\mJ}{\mathbf{J}}
\newcommand{\mX}{\mathbf{X}}
\newcommand{\mS}{\mathbf{S}}
\newcommand{\mA}{\mathbf{A}}
\newcommand{\unos}{\boldsymbol{1}}
\newcommand{\xbarnv}{\bar{\mathbf{x}}_n}
\newcommand{\abs}[1]{\left\vert #1 \right\vert}
\newcommand{\muv}{\boldsymbol{\mu}}
\newcommand{\mcov}{\boldsymbol{\Sigma}}
\newcommand{\vbet}{\boldsymbol{\beta}}
\newcommand{\veps}{\boldsymbol{\epsilon}}
\newcommand{\mC}{\mathbf{C}}
\newcommand{\ceros}{\boldsymbol{0}}
\newcommand{\mH}{\mathbf{H}}
\newcommand{\ve}{\mathbf{e}}
\newcommand{\avec}{\mathbf{a}}
\newcommand{\res}{\textbf{RESPUESTA}\\}
\newcommand{\rojo}[1]{\textcolor{MaterialRed900}{#1}}

\newcommand{\defi}[3]{\textbf{Definición:#3}}
\newcommand{\fin}{$\blacksquare.$}
\newcommand{\finf}{\blacksquare.}

\begin{document}

\begin{table}[ht]
\centering
\begin{tabular}{c}
\textbf{Maestría en Computo Estadístico}\\
\textbf{Inferencia Estadística} \\
\textbf{Tarea 2}\\
\today \\
\emph{Enrique Santibáñez Cortés}\\
Repositorio de Git: \href{https://github.com/Enriquesec/Inferencia_Estad-stica/tree/master/Tareas/Tarea_2}{Tarea 2, IE}.
\end{tabular}
\end{table}

\begin{enumerate}
\def\labelenumi{\arabic{enumi}.}
\setcounter{enumi}{3}
\tightlist
\item
  El siguiente conjuntos de datos contiene mediciones del diámetro de un
  agave, medido en decímetros, en distintas localizaciones no cercanas.
\end{enumerate}

\begin{enumerate}
\def\labelenumi{\alph{enumi})}
\item
  Escriba una función en R que calcule la función de distribución
  empírica para un conjunto de datos dado D. La función debe tomar como
  parámetros al valor \(x\) donde se evalúa y al conjunto de datos
  \(D\). Utilizando esta función grafique la función de distribución
  empírica asociada al conjunto de datos de agave. Ponga atención a los
  puntos de discontinuidad. ¿Qué observa? Nota: Escriba la función
  mediante el algoritmo descrito en las notas de la clase; para este
  ejercicio no vale usar la funciones implementadas en R que hacen lo
  pedido.
\item
  Usando la desigualdad de Dvoretzky-Kiefer-Wolfowitz, escriba una
  función en R que calcule y grafique una región de confianza para la
  función de distribución empírica. La función debe tomar como
  parámetros al conjunto de datos que se usan para contruir la función
  de distribución empírica.
\end{enumerate}

d ) Escriba una función en R que determine el gráfico de probabilidad
normal. La función debe tomar como parámetro al conjunto de datos. ¿Qué
observa? Nota: La misma del inciso a). e) ¿Los datos anteriores se
distribuyen normalmente? Argumente.

\begin{enumerate}
\def\labelenumi{\arabic{enumi}.}
\setcounter{enumi}{4}
\item
  En este ejercicio repasará la estimación de densidades.
\item
  En este ejercicio se comprobará que tan buena es la aproximación dada
  por las reglas empíricas para algunas de las distribuciones estudiadas
  en la clase. Considerese las distribuciones \(Unif(a = −3, b = 3)\),
  \(Normal(0, 1)\), \(Exponencial(2)\), \(Gamma(\alpha= 2, \beta = 1)\),
  \(Gamma(\alpha=3, \beta= 1)\), \(Beta(\alpha= 2, \beta= 2)\),
  \(Weibull(\alpha= 4, \beta= 1)\) y Lognormal\((\mu = 3, \sigma = 2)\).
\end{enumerate}

\begin{enumerate}
\def\labelenumi{\alph{enumi})}
\tightlist
\item
  Para cada una de las distribuciones anteriores, haga una tabla que
  muestre las probabilidades contenidas en los intervalos
  \((\mu − k\sigma, \mu + k\sigma)\), para \(k = 1, 2, 3\). Utilice las
  fórmulas de las medias y varianzas contenidas en las notas para
  determinar \(\mu\) y \(\sigma\) en cada caso. Puede usar R para
  determinar las probabilidades pedidas.
\end{enumerate}


\end{document}
