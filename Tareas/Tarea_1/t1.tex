\documentclass[letterpaper,12pt]{report}
\usepackage{graphicx}
\usepackage{verbatim}
\usepackage[intlimits]{amsmath}
\usepackage{amsbsy}
\usepackage{amsfonts}
\usepackage{colortbl}
\usepackage{multirow}
\usepackage{stmaryrd}
\usepackage[spanish,es-nodecimaldot,es-tabla]{babel}
\usepackage[utf8]{inputenc}
\usepackage{multirow}
\usepackage{cprotect}

\spanishdecimal{.}

\topmargin=-0.7in
\oddsidemargin=0.25in
\evensidemargin=-0.2in
\textwidth=6.4in
% \textheight=8.7in
% \textheight=9.2in
% \footskip=0.25in
% \headheight=15pt

\renewcommand{\textfraction}{0.10}
\renewcommand{\topfraction}{0.90}
\renewcommand{\bottomfraction}{0.90}

\begin{document}
\begin{center}
  {\Large \bf Ciencia de Datos \\
    Tarea 1}
\end{center} 
\vspace{0.3cm}
\begin{center}
  \large Para entregar el 8 de febrero de 2021

  \small La entrega es según las instrucciones dadas al inicio del curso.
\end{center}

\bigskip

\begin{enumerate}
\item Concluye el ejercicio de análisis en los datos \verb|olive| según los objetivos que discutimos en clase, es decir, tratar de identificar las regiones de los aceites y también las áreas dentro de cada región. 
\item Considera los datos que se encuentran en el archivo \verb|suicide_data.csv|, que compila información relacionada con suicidios en 101 países en diferentes años e incluye también algunos indicadores de desarrollo, como el Producto Interno Bruto (GPD) y per capita, Indice de Desarrollo Humano (HDI), entre otra información\footnote{Recopilación de distintas fuentes como United Nations Development Program (2018), World Bank (2018), World Health Organization (2018), entre otros.}.
  \begin{enumerate}
  \item ¿Qué preguntas de interés puedes formular para analizar éste fenómeno en base a estos datos?
  \item \label{i1} Realiza un análisis exploratorio y descriptivo de los datos con las herramientas que consideres apropiadas. ¿Qué estructuras o correlaciones encuentras? Comenta tus hallazgos. 
  \item \label{i2}Trata de responder las preguntas que formulaste en el primer inciso. ¿Puedes obtener conclusiones interesantes?
  \item Considera solamente los datos de México y Estados Unidos. Realiza los incisos \ref{i1} y \ref{i2} por separado y compara los resultados. ¿Qué puedes concluir?
  \item ¿Qué otra información te gustaría tener disponible para analizar éste fenómeno y cómo la utilizarías \cprotect\footnote{Quizá pueda servirte el documento disponible en\\
      \verb|https://www.who.int/mental_health/suicide-prevention/national_strategies_2019/en/| para generar ideas.}?
  \end{enumerate}
\end{enumerate}
  El resultado, para ambos ejercicios, debe ser un reporte corto donde describas los pasos que seguiste y las conclusiones a las que llegas, incluyendo las gráficas más ilustrativas o informativas que tu consideres.

  \textbf{Nota:} La base de datos de suicidios contiene datos faltantes. Indica en el reporte cómo los manejaste.
\end{document}
