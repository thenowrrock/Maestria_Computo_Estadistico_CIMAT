\documentclass[11pt,letterpaper]{article}
\usepackage[utf8]{inputenc}
\usepackage[T1]{fontenc}
\usepackage[spanish]{babel}
\usepackage{amsmath}
\usepackage{amsfonts}
\usepackage{amssymb}
\usepackage{graphicx}
\usepackage{lmodern}
\usepackage{xspace}
\usepackage{multicol}
\usepackage{hyperref}
\usepackage{float}
\usepackage{hyperref}
\usepackage{color}

\hypersetup{colorlinks=true,   linkcolor=MaterialBlue900}
%\usepackage[colorlinks=true, linkcolor=black, urlcolor=blue, pdfborder={0 0 0}]{hyperref}

\usepackage[left=2cm,right=2cm,top=2cm,bottom=2cm]{geometry}
\title{Modelos no paramétricos y de regresión 2018-1}
\author{Tarea examen: pruebas binomiales y tablas de contingencia}
\date{Fecha de entrega: 08/01/2017}
\setlength{\parindent}{0in}
\spanishdecimal{.}

\newcommand{\X}{\mathbb{X}}
\newcommand{\x}{\mathbf{x}}
\newcommand{\Y}{\mathbf{Y}}
\newcommand{\y}{\mathbf{y}}
\newcommand{\xbarn}{\bar{x}_n}
\newcommand{\ybarn}{\bar{y}_n}
\newcommand{\paren}[1]{\left( #1 \right)}
\newcommand{\llaves}[1]{\left\lbrace #1 \right\rbrace}
\newcommand{\barra}{\,\vert\,}
\newcommand{\mP}{\mathbb{P}}
\newcommand{\mE}{\mathbb{E}}
\newcommand{\mR}{\mathbb{R}}
\newcommand{\mJ}{\mathbf{J}}
\newcommand{\mX}{\mathbf{X}}
\newcommand{\mS}{\mathbf{S}}
\newcommand{\mA}{\mathbf{A}}
\newcommand{\unos}{\boldsymbol{1}}
\newcommand{\xbarnv}{\bar{\mathbf{x}}_n}
\newcommand{\abs}[1]{\left\vert #1 \right\vert}
\newcommand{\muv}{\boldsymbol{\mu}}
\newcommand{\mcov}{\boldsymbol{\Sigma}}
\newcommand{\vbet}{\boldsymbol{\beta}}
\newcommand{\veps}{\boldsymbol{\epsilon}}
\newcommand{\mC}{\mathbf{C}}
\newcommand{\ceros}{\boldsymbol{0}}
\newcommand{\mH}{\mathbf{H}}
\newcommand{\ve}{\mathbf{e}}
\newcommand{\avec}{\mathbf{a}}
\newcommand{\res}{\textbf{RESPUESTA}\\}
\newcommand{\rojo}[1]{\textcolor{MaterialRed900}{#1}}

\newcommand{\defi}[3]{\textbf{Definición:#3}}
\newcommand{\fin}{$\blacksquare.$}
\newcommand{\finf}{\blacksquare.}
\newcommand{\tr}{\text{tr}}
\newcommand*{\temp}{\multicolumn{1}{r|}{}}

\newcommand{\grstep}[2][\relax]{%
   \ensuremath{\mathrel{
       {\mathop{\longrightarrow}\limits^{#2\mathstrut}_{
                                     \begin{subarray}{l} #1 \end{subarray}}}}}}
\newcommand{\swap}{\leftrightarrow}

\begin{document}
\begin{table}[ht]
\centering
\begin{tabular}{c}
\textbf{Maestría en Computo Estadístico}\\
\textbf{Álgebra Matricial} \\
\textbf{Tarea 3}\\
\today \\
\emph{Enrique Santibáñez Cortés}\\
Repositorio de Git: \href{https://github.com/Enriquesec/Algebra_matricial/tree/master/tareas/Tarea_5}{Tarea 5, AM}.
\end{tabular}
\end{table}
Todos los cálculos deben ser a mano.
\begin{enumerate}

%Problema 1
%------------------------------------------------------------------------------------------------------%
%------------------------------------------------------------------------------------------------------%
%------------------------------------------------------------------------------------------------------%
\item Encuentre la descomposición $LDU$ de la matriz
\begin{align*}
\begin{pmatrix}
1 & 2 & 4\\
3 & 8 & 14\\
2 & 6 & 13
\end{pmatrix}.
\end{align*}

%Problema 2
%------------------------------------------------------------------------------------------------------%
%------------------------------------------------------------------------------------------------------%
%------------------------------------------------------------------------------------------------------%
\item Sea 
\begin{align*}
A=\begin{pmatrix}
1 & 2 & 3\\
2 & 4 & 5\\
1 & 3 & 4
\end{pmatrix}.
\end{align*}
Observe que $A$ es no singular y siempre tiene descomposición $PLU$. Pruebe que, sin embargo, A no tiene descomposición $LU$. (Sugerencia: no use eliminación, use un teorema.)

%Problema 3
%------------------------------------------------------------------------------------------------------%
%------------------------------------------------------------------------------------------------------%
%------------------------------------------------------------------------------------------------------%
\item Sea 
\begin{align*}
A=\begin{pmatrix}
1 & 1 & 2 \\
1 & 3 & 8\\
2 & 8 & 23
\end{pmatrix}
\end{align*}
Encuentre la descomposición $LDL^t$ de $A$. ¿Es $A$ positiva definida? Explique. En tal caso encuentre su descomposición de Cholesky.

%Problema 4
%------------------------------------------------------------------------------------------------------%
%------------------------------------------------------------------------------------------------------%
%------------------------------------------------------------------------------------------------------%
\item Sea $A$ la matriz por bloques
\begin{align*}
A=\begin{pmatrix}
B & C \\
0 & E
\end{pmatrix}
\end{align*}
con $B$ y $E$ no singulares. Demuestre que $A^{-1}$ es de la forma
\begin{align*}
A=\begin{pmatrix}
B^{-1}& X \\
0 & E^{-1}
\end{pmatrix}
\end{align*}
y encuentre $X$. Luego, si 
\begin{align*}
A_1=\begin{pmatrix}
B & 0 \\
D & E
\end{pmatrix}
\end{align*}
y encuentre $Y$.

%Problema 5
%------------------------------------------------------------------------------------------------------%
%------------------------------------------------------------------------------------------------------%
%------------------------------------------------------------------------------------------------------%
\item Sea $F$ un matriz fija de $3\times 2$ y sea $$H=\{A\in M_{2\times 4}(\mathbb{R})|FA=0\}.$$
Determine si $H$ es un subespacio de $M_{2\times 4}(\mathbb{R})$.

%Problema 6
%------------------------------------------------------------------------------------------------------%
%------------------------------------------------------------------------------------------------------%
%------------------------------------------------------------------------------------------------------%
\item Demuestre que en $\mR^2$ los únicos subespacioes posibles son $\{0\}$, las líneas que pasan por el origen y $\mR^2$. Enuncie y demuestre un resultado análogo para $\mR^3$.

%Problema 7
%------------------------------------------------------------------------------------------------------%
%------------------------------------------------------------------------------------------------------%
%------------------------------------------------------------------------------------------------------%
\item Sea $S\subset \mR^3$ dado por 
\begin{align*}
S=\left\{\begin{pmatrix}
2\\
-1\\
1
\end{pmatrix}, \begin{pmatrix}
2\\
-3\\
2
\end{pmatrix} \right\}.
\end{align*}
Determine si \begin{align*}
\begin{pmatrix}
-2\\
-3\\
1
\end{pmatrix}
\end{align*}
esta en $\text{gen}(S)$, y si \begin{align*}
\begin{pmatrix}
-8\\
5\\
4
\end{pmatrix}
\end{align*}
esta en $\text{gen}(S)$.


%Problema 8
%------------------------------------------------------------------------------------------------------%
%------------------------------------------------------------------------------------------------------%
%------------------------------------------------------------------------------------------------------%
\item Sea $V$ un espacio vectorial y $W, \ Z$ subespacios de $V$. Al definir el espacio $W+Z$ no se hizo distinción en el orden. ¿Por qué no?, es decir, ¿es cierto que
$W + Z = Z + W$? Argumente su respuesta. Enuncie y demuestre un resultado similar para cualquier número finito de subespacios.

%Problema 9
%------------------------------------------------------------------------------------------------------%
%------------------------------------------------------------------------------------------------------%
%------------------------------------------------------------------------------------------------------%
\item Encuentre $A$ tal que $W=\mathcal{C}(A)$, donde
\begin{align*}
W=\left\{\left.\begin{pmatrix}
r-s\\
2r+3t\\
r+3s-3t\\
s+t
\end{pmatrix}\right| r,s,t\in \mR \right\}.
\end{align*}

%Problema 10
%------------------------------------------------------------------------------------------------------%
%------------------------------------------------------------------------------------------------------%
%------------------------------------------------------------------------------------------------------%
\item Sea \begin{align*}
A=\begin{pmatrix}
 1 & -1 & 6 & 0\\
10 & -8 &-2 &-2\\
 0 &  2 & 2 &-2\\
 1 &  1 & 0 &-2
\end{pmatrix}
\end{align*}.
Encuentre un vector en $\mathcal{N}$ (A). Encuentre dos vectores distintos (que no sean múltiplos) en $\mathcal{C}(A)$. ¿Se pueden encontrar más vectores en $\mathcal{N}(A)$ y $\mathcal{C}(A)$, respectivamente, a los ya encontrados que no sean combinación lineal de los anteriores?





\end{enumerate}
\end{document}